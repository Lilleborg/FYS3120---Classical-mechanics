\documentclass[11pt,a4paper]{report}

\usepackage[utf8]{inputenc}
\usepackage[T1]{fontenc}

\pagestyle{empty}

\usepackage{graphicx} % Include figure files
\usepackage{amstext,amsbsy,amssymb}
%\usepackage{times} 

%% Numbered problems
\newcounter{excount}[chapter]
\newenvironment{exercise}[1][]{\addtocounter{excount}{1} \noindent {\bf Problem
    \arabic{excount} \ \ #1}\hspace{2mm}}{\vspace{4mm}}


\title{FYS3120 Classical Mechanics and Electrodynamics\\ 
\vspace{15mm}Problem set 8}


%%%%%%%
\begin{document}
%%%%%%%
\maketitle


%%%%%%%%%%
\begin{exercise}
%%%%%%%%%%
A monocromatic light source is at rest in the laboratory and sends photons with frequency $\nu_0$ towards a mirror which has its reflective surface perpendicular to the beam direction. The mirror moves away from the light source with velocity $v$. Use the transformation formula for four-momentum $p^\mu=(E/c,\vec p)$ and the Planck relation $E=h\nu$ to:
\begin{itemize}
\item[\bf a)] Find the frequency of the emitted and reflected light in the rest frame of the mirror.
\item[\bf b)] Find the frequency of reflected light in the lab system.
\end{itemize}
\end{exercise}


%%%%%%%%%%
\begin{exercise}
%%%%%%%%%%
Figure~\ref{fig:collisionA} shows a particle with mass $m$ and (relativistic) kinetic energy $K$ in the laboratory frame $S$. The particle is moving towards another particle, with the same mass $m$, which is at rest in $S$.

%%%%%%%%%%%%
\begin{figure}[h]
\begin{center}
\includegraphics[width=6cm]{CollisionA.eps}
\end{center}
\caption{Collision between two particles of mass $m$ resulting in a particle with mass $M$.}
\label{fig:collisionA}
\end{figure}
%%%%%%%%%%%%

\begin{itemize}
\item[\bf a)] Find the velocity $v$ of the first particle expressed in terms of the dimensionless quantity $\alpha=K/mc^2$ (and the speed of light).
\end{itemize}

First we will assume that the particles collide in  such a way that they form one particle after the collision (a totally inelastic collision).
\begin{itemize}
\item[\bf b)] Determine the compound particle's energy $E$, momentum $P$, velocity $V$ and mass $M$ in terms of the velocity $v$ (or $\gamma$). Find the change in the total kinetic energy of the system due to the collision.
\end{itemize}

In the rest of the exercise we will assume that the situation before the collision is as described earlier, but that the particles now collide elastically, {\it i.e.}\ after the collision the two particles are the same as before the collision, with no change in their masses. The collision happens in such a way that the particles after the collision make the same angle, $\theta$, with the $x$-axis in the lab frame $S$, see Fig.~\ref{fig:CollisionB}.

%%%%%%%%%%%%
\begin{figure}[h]
\begin{center}
\includegraphics[width=6cm]{CollisionB.eps}
\end{center}
\caption{Collision between two particles of mass $m$ resulting in scattering at angle $\theta$.}
\label{fig:CollisionB}
\end{figure}
%%%%%%%%%%%%

\begin{itemize}
\item[\bf c)] Show that after the collision the particles have the same magnitude of momentum ($|\vec p_1|=|\vec p_2|$) and energy ($E_1=E_2$).
\item[\bf d)] Determine $E\equiv E_1=E_2$ and $p\equiv |\vec p_1|=|\vec p_2|$.
\item[\bf e)] Determine the angle $\theta$. Find $\theta$ in the limiting cases when $\alpha=K/mc^2$ goes to zero and to infinity. Show that $\theta<\pi/4$. 
\end{itemize}
\end{exercise}



%%%%%%%%
 \end{document}
 %%%%%%%%


