\documentclass[11pt,a4paper]{report}

\usepackage[utf8]{inputenc}
\usepackage[T1]{fontenc}

%%%%%%%%%%% Own packages
\usepackage[a4paper, margin=1in]{geometry}
\usepackage{multicol}

% Boxing
\usepackage[most]{tcolorbox}
\tcbset{enhanced}
%\usepackage{capt-of}	% captions in boxes

% Header/footer
\usepackage{fancyhdr}
\pagestyle{fancy}
\renewcommand{\headrulewidth}{0pt}

% Listsings and items
\usepackage[shortlabels]{enumitem}
\setenumerate{wide,labelwidth=!, labelindent=0pt}
\usepackage{varioref}
\usepackage{hyperref}
\usepackage{cleveref}

% Maths
\usepackage{physics}
\usepackage{cancel}
\usepackage{amstext,amsbsy,amssymb}
\usepackage{mathtools}
\usepackage{times} 
\usepackage{siunitx}
\usepackage{tensor}

%% Graphics
\usepackage{caption}
\captionsetup{margin=10pt,font=small,labelfont=bf}
%\renewcommand{\thesubfigure}{(\alph{subfigure})} % Style: 1(a), 1(b)
%\pagestyle{empty}
\usepackage{graphicx}% Include figure files
%\usepackage{epstopdf}
%\usepackage{pdfsync}


%% Numbered problems
\newcounter{excount}[chapter]
\newenvironment{exercise}[1][]{\addtocounter{excount}{1} \noindent {\bf Problem
    \arabic{excount} \ \ #1}\hspace{2mm}}{\vspace{4mm}}

%% Solution environment
\newenvironment{solution}[1][]
    {\begin{tcolorbox}[title=Solution #1,halign lower=right,breakable]
    }
    {
    \tcblower Jakob Borg
    \end{tcolorbox}
	\vspace{5mm}
    }

%% Figure command
\newcommand{\fig}[2][]
{
\begin{center}
\includegraphics[width=0.49\linewidth]{#2}
\captionof{figure}{#1}
\label{fig:fig\arabic{figure}}
\end{center}
}

%% Quic-half
\newcommand{\half}
{
\frac{1}{2}
}

%% Quick center-of-mass
\newcommand{\com}
{
\text{CoM}
}

%% quad while
\newcommand{\qwhile}
{
\qq{while}
}

%% quad where
\newcommand{\qwhere}
{
\qq{where}
}

%% Quick transformation arrow
\newcommand{\transarrow}
{
\quad \rightarrow \quad
}

%% Lagrangian's equation
\newcommand{\Leq}[1]
{
\dv{t}\qty(\pdv{L}{\dot{#1}}) - \pdv{L}{#1}
}

%% Quick L and H pdv
\newcommand{\Lpdv}[1]
{
\pdv{L}{#1}
}
\newcommand{\Hpdv}[1]
{
\pdv{H}{#1}
}

%% Time derivative of Lpdv
\newcommand{\dvtLpdv}[1]
{
\dv{t} \qty(\Lpdv{#1}) 
}

%% Time derivative of argument
\newcommand{\dvt}[1]
{
\dv{t} \qty(#1)
}

%% Quick theta dot and phi dot
\newcommand{\dtheta}
{
\dot{\theta}
}
\newcommand{\dphi}
{
\dot{\phi}
}
\newcommand{\ddtheta}
{
\ddot{\theta}
}
\newcommand{\ddphi}
{
\ddot{\phi}
}

%% Quick dot vector
\newcommand{\dotva}[1]
{
\dot{\va{#1}}
}
\newcommand{\ddotva}[1]
{
\ddot{\va{#1}}
}

%% Quick sin cos of theta and phi commands
\newcommand{\cost}
{
\cos(\theta)
}
\newcommand{\sint}
{
\sin(\theta)
}
\newcommand{\coscost}
{
\cos[2](\theta)
}
\newcommand{\sinsint}
{
\sin[2](\theta)
}
\newcommand{\cosp}
{
\cos(\phi)
}
\newcommand{\sinp}
{
\sin(\phi)
}

%% Quick sin cos of omega t
\newcommand{\coswt}
{
\cos(\omega t)
}
\newcommand{\sinwt}
{
\sin(\omega t)
}
\newcommand{\coscoswt}
{
\cos^2(\omega t)
}
\newcommand{\sinsinwt}
{
\sin^2(\omega t)
}

\title{FYS3120 Classical Mechanics and Electrodynamics\\ 
\vspace{15mm}Problem set 8}
\author{Jakob Borg}
%%%%%%%
\begin{document}
%%%%%%%

\maketitle

%\lhead{Jakob Borg}
\lhead{Problem set 8 FYS3120}
\rhead{Jakobbor}
%%%%%%%%

%%%%%%%%%%
\begin{exercise}
%%%%%%%%%%
A monocromatic light source is at rest in the laboratory and sends photons with frequency $\nu_0$ towards a mirror which has its reflective surface perpendicular to the beam direction. The mirror moves away from the light source with velocity $v$. Use the transformation formula for four-momentum $p^\mu=(E/c,\vec p)$ and the Planck relation $E=h\nu$ to:
\begin{itemize}
\item[\bf a)] Find the frequency of the emitted and reflected light in the rest frame of the mirror.
\begin{solution}[\bf 1.a]
Assuming the mirror and the light moving in the x-direction for simplicity, $\vu{v} = \vu{x}$. In the lab frame, RF $S$, the light has four-momentum $p^\mu$, which transforms to $p^{'\mu}$ in the mirror's frame, RF $S^{'}$, as follows
\begin{align*}
p^\mu &= \qty(\frac{E}{c},\va{p}) \quad \rightarrow\quad p\indices{^'^\rho} = L\indices{^\rho_\mu}p^\mu
\end{align*}
where 
\begin{align*}
p^0 &= \frac{E}{c}= \frac{h\nu_0}{c}, & m = 0 \Rightarrow E = pc \Rightarrow  \va{p} &= \qty(\frac{h\nu_0}{c} ,\, 0,\, 0), & \gamma &= \frac{1}{\sqrt{1-\beta^2}}, & \beta &= \frac{v}{c}.
\end{align*}
Notice that $p^0 = p^1$. Transforming each component I find
\begin{align*}
p\indices{^'^0} &= \gamma\qty(p^0-\beta p^1) = \gamma \qty(\frac{h\nu_0}{c}-\beta\frac{h\nu_0}{c}) = \frac{h\nu_0}{c}\gamma (1-\beta)
\\
\qq{similarly} p\indices{^'^1} &= \gamma\qty(p^1-\beta p^0) = \frac{h\nu_0}{c}\gamma(1-\beta)
\\
p\indices{^'^2} &= p^2 \qand p\indices{^'^3} = p^3
\end{align*}
The relationship between the frequencies in the two frames can be found from the relation between the $p^0$s
\begin{align*}
p\indices{^'^0} &= \frac{h\nu^{'}_0}{c} = \frac{h\nu_0}{c}\gamma (1-\beta)
\\
\qq*{so:} \nu_0^{'} &= \nu_0 \gamma(1-\beta)
\end{align*}
\end{solution}
\item[\bf b)] Find the frequency of reflected light in the lab system.
\begin{solution}[\bf 1.b]
I find the frequency in the lab system by transforming back into the lab RF
\begin{align*}
p^\nu &= L\indices{^\nu_\rho}p\indices{^'^\rho}.
\end{align*}
Here one needs to mind the sign in the velocity, as the lab frame moves in the opposite direction of the mirror, $\beta = -v/c$. I only need the zero'th component to find the frequency
\begin{align*}
p^0 &= \gamma \qty(p\indices{^'^0}-\beta p\indices{^'^1}) = \gamma \qty( \frac{h\nu_0}{c}\qty(1-\beta)-\beta \frac{h\nu_0}{c}\qty(1-\beta))
\\
&= \frac{h\nu_0}{c}\gamma^2\qty(1-\beta)^2 \qq{using} \gamma^2 = \frac{1}{(1-\beta)(1+\beta)}
\\
&= \frac{h\nu_0}{c}\frac{1-\beta}{1+\beta}.
\end{align*}
Let the expression for $p^0$ be $\frac{h\nu_R}{c}$ for the received frequency $\nu_R$, then
\begin{align*}
\nu_R &= \nu_0\frac{1-\beta}{1+\beta} \qwhere \beta = -\frac{v}{c}
\end{align*}
\end{solution}
\end{itemize}
\end{exercise}


%%%%%%%%%%
\begin{exercise}
%%%%%%%%%%
Figure~\ref{fig:collisionA} shows a particle with mass $m$ and (relativistic) kinetic energy $K$ in the laboratory frame $S$. The particle is moving towards another particle, with the same mass $m$, which is at rest in $S$.

%%%%%%%%%%%%
\begin{figure}[h]
\begin{center}
\includegraphics[width=6cm]{CollisionA.eps}
\end{center}
\caption{Collision between two particles of mass $m$ resulting in a particle with mass $M$.}
\label{fig:collisionA}
\end{figure}
%%%%%%%%%%%%

\begin{itemize}
\item[\bf a)] Find the velocity $v$ of the first particle expressed in terms of the dimensionless quantity $\alpha=K/mc^2$ (and the speed of light).
\end{itemize}
\begin{solution}[\bf 2.a]
The relativistic kinetic energy is the relativistic energy minus the rest mass energy. Let $m_1$ be the mass of particle 1.
\begin{align*}
K &= E - m_1c^2 = \gamma m_1c^2 - m_1 c^2 = m_1c^2(\gamma-1)
\\
\frac{K}{m_1c^2} = \alpha &= \gamma -1 \quad \Rightarrow \frac{1}{\sqrt{1-\beta^2}} = \alpha+1
\\
1-\beta^2 &= \frac{1}{(\alpha+1)^2}
\\
\beta &= \sqrt{1-\frac{1}{(\alpha+1)^2}} \quad \Rightarrow v = c\sqrt{1-\frac{1}{(\alpha+1)^2}}
\end{align*}
\end{solution}

First we will assume that the particles collide in  such a way that they form one particle after the collision (a totally inelastic collision).
\begin{itemize}
\item[\bf b)] Determine the compound particle's energy $E$, momentum $P$, velocity $V$ and mass $M$ in terms of the velocity $v$ (or $\gamma$). Find the change in the total kinetic energy of the system due to the collision.
\end{itemize}
\begin{solution}[\bf 2.b]
To determine the properties of the compound particle I use conservation of relativistic energy and momentum through the four-momentum, which conserves the two quantities separately
\begin{align*}
\sum_i p_i^\mu &= \sum_f p_f^\mu
\end{align*}
where index $i$ and $f$ describes the initial and final quantities before and after the collision. In these calculations I constantly use that $\va{v}_2 = 0$ and $m_1 = m_2$.

First I find the final energy, $E_f = E = \gamma_f M c^2$, through the zero'th component of $p^\mu$:
\begin{align}
E &= \gamma_f M c^2 = \gamma_1 m_1 c^2 + \gamma_2 m_2 c^2 \qwhere \gamma_2 = 1 \notag
\\
\Rightarrow E &= \qty(\gamma_1+1)m_1 c^2 \label{eq:E}
\end{align}
The momentum is easily found from the vector component of $p^\mu$:
\begin{align}
\va{P} &= \gamma_f M \va{V} = \gamma_1 m_1 \va{v}_1 + \gamma_2 m_2 \va{v}_2 = \gamma_1 m_1 \va{v}_1 \notag
\\
\Rightarrow \va{P} &= \gamma_1 m_1 \va{v}_1 \label{eq:P}
\end{align}
Using the two last results, \cref{eq:E,eq:P}, I find the velocity:
\begin{align}
\va{P} &= \va{p}_i \notag
\\
\gamma_f M \va{V} &= \gamma_1 m_1 \va{v}    \notag
\\
\va{V} &= \frac{\gamma_1m_1\va{v}_1}{\gamma_f M} \qq{from \cref{eq:E}} \gamma_f M = (\gamma_1+1)m_1 \notag
\\
\Rightarrow \va{V} &= \frac{\gamma_1}{\gamma_1+1}\va{v}_1 \label{eq:V}
\end{align}
Rewriting the relativistic energies as $E = \sqrt{m^2c^4+p^2c^2}$, using the result from \cref{eq:P} to express the compound energy and $E_2 = m_2c^2$ as the rest energy of body 2 I find
\begin{align}
E_f = E_1 + E_2 &= \sqrt{M^2c^4+\va{P}^2c^2} = \sqrt{m_1^2c^4+\va{p}_1^2m_1^2c^2} + m_2c^2 \notag
\\
\sqrt{M^2c^4 + \qty(\gamma_1m_1\va{v}_1)^2c^2} &= \sqrt{m_1^2c^4 + \qty(\gamma_1m_1\va{v}_1)^2c^2} + m_2c^2    \notag
\\
M^2c^4 +\cancel{\qty(\gamma_1m_1\va{v})^2c^2} &= m_1^2c^4+\cancel{\gamma_1^2m_1^2\va{v}_1c^2}+ 2m_2c^2\sqrt{m_1^2c^4 + \qty(\gamma_1m_1\va{v}_1)^2c^2} + m_2^2c^4    \notag
\\
m_1 = m_2 \qc M^2c^4 &= 2m_1^2c^4 + 2m_1c^2\sqrt{m_1^2c^4 + \qty(\gamma_1m_1\va{v}_1)^2c^2} \eval \cdot \frac{1}{2mc^2}   \notag
\\
\frac{M^2c^2}{2m_1} &= m_1c^2 + \sqrt{m_1^2c^4 + \qty(\gamma_1m_1\va{v}_1)^2c^2}     \notag
\\
&= E_2 + E_1 = \qty(\gamma_1+1)m_1c^2    \notag
\\
\Rightarrow M &= m\sqrt{2\qty(\gamma_1+1)} \label{eq:M}
\end{align}
\end{solution}

In the rest of the exercise we will assume that the situation before the collision is as described earlier, but that the particles now collide elastically, {\it i.e.}\ after the collision the two particles are the same as before the collision, with no change in their masses. The collision happens in such a way that the particles after the collision make the same angle, $\theta$, with the $x$-axis in the lab frame $S$, see Fig.~\ref{fig:CollisionB}.

%%%%%%%%%%%%
\begin{figure}[h]
\begin{center}
\includegraphics[width=6cm]{CollisionB.eps}
\end{center}
\caption{Collision between two particles of mass $m$ resulting in scattering at angle $\theta$.}
\label{fig:CollisionB}
\end{figure}
%%%%%%%%%%%%

\begin{itemize}
\item[\bf c)] Show that after the collision the particles have the same magnitude of momentum ($|\vec p_1|=|\vec p_2|$) and energy ($E_1=E_2$).
\begin{solution}[\bf 2.c]
To avoid confusion with the indices I will write the quantities before the collision with a bar.

I show $|\va{ p}_1|=|\va {p}_2| = p$ by conservation of momentum
\begin{align*}
\bar{\va{p}}_1 &= \va{p}_1+\va{p}_2
\end{align*}
writing out the equation on component form, as the collision is elastic these can be solved separately;
\begin{align}
\bar{p}_1\qty(1,0,0) &= p_1\qty(\cost,\sint,0) + p_2\qty(\cost,-\sint,0) \label{eq:P_conserved_component}
\end{align}
reveals the relation in the second component
\begin{align*}
\bar{p}_1\cdot 0 = 0 &= p_1\sint - p_2\sint = \sint (p_1-p_2)
\\
\Rightarrow p_1 &= p_2
\end{align*}
where $\theta=0$ is not considered.

The energies after the collision can be written as 
\begin{equation}
\begin{aligned}
E_1^2 &= p_1^2 c^2+m_1^2c^4
\\
E_2^2 &= p_2^2 c^2+m_2^2c^4. \label{eq:Energies_after}
\end{aligned}
\end{equation}
As $m_1=m_2$ and I've shown $|\va{p}_1|=|\va{p}_2| = p_1 =  p_2$ these energies are identical.

This problem can also be solved (maybe in a nicer way) by looking at the system from the center-of-mass RF (CoM).
\end{solution}

\item[\bf d)] Determine $E\equiv E_1=E_2$ and $p\equiv |\vec p_1|=|\vec p_2|$.
\begin{solution}[\bf 2.d]
To determine the magnitude of momentum $p$ I use the first component of \cref{eq:P_conserved_component} and that the magnitude of the momenta after the collision are equal. Having the expression for $p$ I can use \cref{eq:Energies_after} to express the energy by the same argument as in \textbf{2.c}:
\begin{align*}
\bar{p} &= 2p\cost
\\
\Rightarrow p &= \frac{\bar{p}}{2\cost} \qc E = \sqrt{\frac{\bar{p}^2c^2}{4\coscost} + m^2c^4}
\end{align*}
where I have stopped using subscript $1$ and $2$ as all the quantities discussed here are equal for both particles.
\end{solution}

\item[\bf e)] Determine the angle $\theta$. Find $\theta$ in the limiting cases when $\alpha=K/mc^2$ goes to zero and to infinity. Show that $\theta<\pi/4$. 
\end{itemize}

\begin{solution}[\bf 2.e]
To determine the angle I will look at the conserved energy. In this calculation there are a lot of algebra and substitutions. Other approaches probably exists where the answer is easier to interpret, but this is the one I've found:
\begin{align*}
2E&= \bar{E}_1 + \bar{E}_2 
\\
2 \sqrt{\frac{\bar{p}^2c^2}{4\coscost} + m^2c^4} &= \sqrt{m^2c^4+\bar{p}^2c^2}+mc^2 \qq{square both}
\\
4\qty( \frac{\bar{p}^2c^2}{4\coscost} + m^2c^4 ) &= m^2c^4+\bar{p}^2c^2 + 2mc^2\sqrt{m^2c^4+\bar{p}^2c^2} + m^2c^4
\\
\frac{\bar{p}^2c^2}{\coscost} - \bar{p}^2c^2&= 2mc^2 \qty(\sqrt{m^2c^4+\bar{p}^2c^2} - mc^2) \tag{*} \label{eq:*}
\end{align*}
where in the last step I gathered all the terms with $mc^2$ on the right hand side, which contains the kinetic energy:
\begin{align*}
2mc^2 \qty(\sqrt{m^2c^4+\bar{p}^2c^2} - mc^2) &= 2mc^2\bar{K}
\\
&\qq*{where} \bar{K} = mc^2\qty(\bar{\gamma}_1-1) \qq{from \textbf{2.a}}
\\
=& 2 m^2c^4\qty(\bar{\gamma}_1-1).
\end{align*}
Rewriting \cref{eq:*} using the kinetic energy and $\bar{p}^2=\bar{\gamma}_1^2m^2\bar{v}_1^2$ I find
\begin{align*}
\bar{\gamma}_1^2m^2\bar{v}_1^2c^2\qty(\frac{1}{\coscost} - 1)&= 2 m^2c^4\qty(\bar{\gamma}_1-1)
\\
\bar{\gamma}_1^2\bar{\beta}^2\qty(\frac{1}{\coscost} - 1) &= 2\qty(\bar{\gamma}_1-1)
\\
\Rightarrow \coscost &= \frac{1}{\frac{2\qty(\bar{\gamma}_1-1)}{\bar{\gamma}_1^2\bar{\beta}^2}+1} = \frac{\bar{\gamma}_1^2\bar{\beta}^2}{\bar{\gamma}_1^2\bar{\beta}^2 +2\qty(\bar{\gamma}_1-1) }.
\end{align*}
where $\bar{\beta} = \frac{\bar{v}_1}{c}$. I have now expressed the angle using only the velocity of body 1 before the collision, so to reduce clutter I will stop using the bar notation. With $\gamma_1 = \frac{1}{\sqrt{1-\beta^2}}$ I find an expression for the angle $\theta$:
\begin{align*}
\coscost &= \frac{\beta^2}{\qty(1-\beta^2)\qty[\frac{\beta^2}{1-\beta^2}+\frac{2}{\sqrt{1-\beta^2}}-2]}
\\
&= \frac{\beta^2}{\beta^2 + 2\sqrt{1-\beta^2}-2\qty(1-\beta^2)}.
\end{align*}
For $\alpha \rightarrow \infty$ the kinetic energy is infinite and so $\beta \rightarrow 1$ as the velocity approaches $c$. Then
\begin{align*}
\lim_{\beta\rightarrow 1} \coscost &= \lim_{\beta\rightarrow 1}  \frac{\beta^2}{\beta^2 + 2\sqrt{1-\beta^2}-2\qty(1-\beta^2)} = 1
\\
\theta &= \arccos(\sqrt{1}) = 0.
\end{align*}

In the limit $\alpha \rightarrow 0$ the kinetic energy is zero, that means $\beta = \frac{v}{c} \rightarrow 0$. To find $\theta$ in the limits I use L'Hôpital's rule as I have a $0/0$ expression:
\begin{align*}
\lim_{\beta\rightarrow 0} \coscost &\overset{\mathclap{\mathrm{(H)}}}{=} \lim_{\beta\rightarrow 0} \frac{2\beta}{2\beta -2\beta \qty(1-\beta^2)^{-1/2} + 4\beta}
\\
&=\lim_{\beta\rightarrow 0}  \frac{2\beta}{6\beta - 2\beta \qty(1-\beta^2)^{-1/2} } = \lim_{\beta\rightarrow 0} \frac{2}{6-\frac{2}{\sqrt{1-\beta^2}}} = \frac{1}{2}
\\
\theta &= \arccos(\frac{1}{\sqrt{2}}) = \frac{\pi}{4}.
\end{align*}
As $0 < \alpha < \infty$ I have that $\theta < \frac{\pi}{4}$ as I wanted to show.
\end{solution}
\end{exercise}



%%%%%%%%
 \end{document}
 %%%%%%%%


