\documentclass[11pt,a4paper]{report}

\usepackage[utf8]{inputenc}
\usepackage[T1]{fontenc}

%%%%%%%%%%% Own packages
\usepackage[a4paper, margin=1in]{geometry}
\usepackage{multicol}

% Boxing
\usepackage[most]{tcolorbox}
\tcbset{enhanced}
%\usepackage{capt-of}	% captions in boxes

% Header/footer
\usepackage{fancyhdr}
\pagestyle{fancy}
\renewcommand{\headrulewidth}{0pt}

% Listsings and items
\usepackage[shortlabels]{enumitem}
\setenumerate{wide,labelwidth=!, labelindent=0pt}
\usepackage{varioref}
\usepackage{hyperref}
\usepackage{cleveref}

% Maths
\usepackage{physics}
\usepackage{cancel}
\usepackage{amstext,amsbsy,amssymb}
\usepackage{times} 
\usepackage{siunitx}

%% Graphics
\usepackage{caption}
\captionsetup{margin=10pt,font=small,labelfont=bf}
%\renewcommand{\thesubfigure}{(\alph{subfigure})} % Style: 1(a), 1(b)
%\pagestyle{empty}
\usepackage{graphicx}% Include figure files
%\usepackage{epstopdf}
%\usepackage{pdfsync}


%% Numbered problems
\newcounter{excount}[chapter]
\newenvironment{exercise}[1][]{\addtocounter{excount}{1} \noindent {\bf Problem
    \arabic{excount} \ \ #1}\hspace{2mm}}{\vspace{4mm}}

%% Solution environment
\newenvironment{solution}
    {\begin{tcolorbox}[title=Solution,halign lower=right,breakable]
    }
    {
    \tcblower Jakob Borg
    \end{tcolorbox}
	\vspace{5mm}
    }

%% Figure command
\newcommand{\fig}[2][]
{
\begin{center}
\includegraphics[width=0.49\linewidth]{#2}
\captionof{figure}{#1}
\label{fig:fig\arabic{figure}}
\end{center}
}

%% Quic-half
\newcommand{\half}
{
\frac{1}{2}
}

%% Quick center-of-mass
\newcommand{\com}
{
\text{c.m.}
}

%% quad while
\newcommand{\qwhile}
{
\qq{while}
}

%% quad where
\newcommand{\qwhere}
{
\qq{where}
}

%% Lagrangian's equation
\newcommand{\Leq}[1]
{
\dv{t}\qty(\pdv{L}{\dot{#1}}) - \pdv{L}{#1}
}

%% Quick L and H pdv
\newcommand{\Lpdv}[1]
{
\pdv{L}{#1}
}
\newcommand{\Hpdv}[1]
{
\pdv{H}{#1}
}

%% Time derivative of Lpdv
\newcommand{\dvtLpdv}[1]
{
\dv{t} \qty(\Lpdv{#1}) 
}

%% Time derivative of argument
\newcommand{\dvt}[1]
{
\dv{t} \qty(#1)
}

%% Quick theta dot and phi dot
\newcommand{\dtheta}
{
\dot{\theta}
}
\newcommand{\dphi}
{
\dot{\phi}
}
\newcommand{\ddtheta}
{
\ddot{\theta}
}
\newcommand{\ddphi}
{
\ddot{\phi}
}

%% Quick dot vector
\newcommand{\dotva}[1]
{
\dot{\va{#1}}
}
\newcommand{\ddotva}[1]
{
\ddot{\va{#1}}
}

%% Quick sin cos of theta and phi commands
\newcommand{\cost}
{
\cos(\theta)
}
\newcommand{\sint}
{
\sin(\theta)
}
\newcommand{\coscost}
{
\cos[2](\theta)
}
\newcommand{\sinsint}
{
\sin[2](\theta)
}
\newcommand{\cosp}
{
\cos(\phi)
}
\newcommand{\sinp}
{
\sin(\phi)
}

%% Quick sin cos of omega t
\newcommand{\coswt}
{
\cos(\omega t)
}
\newcommand{\sinwt}
{
\sin(\omega t)
}
\newcommand{\coscoswt}
{
\cos^2(\omega t)
}
\newcommand{\sinsinwt}
{
\sin^2(\omega t)
}

\title{FYS3120 Classical Mechanics and Electrodynamics\\ 
\vspace{15mm}Problem set 5}
\author{Jakob Borg}
%%%%%%%
\begin{document}
%%%%%%%

\maketitle

%\lhead{Jakob Borg}
\lhead{Problem set 5 FYS3120}
\rhead{Jakobbor}
%%%%%%%%

%%%%%%%%
\begin{exercise}
A particle with mass $m$ moves freely on a horizontal plane. There are no constraints on the motion, but in the following we will consider the free motion described in a rotating reference frame. We refer to the Cartesian coordinates of a fixed frame as $(x,y)$ and the coordinates of the rotating frame as $(\xi,\eta$). They are related by the standard expressions
\begin{eqnarray}
&&x=\xi\cos\omega t-\eta\sin\omega t, \\
&&y=\xi\sin\omega t+\eta\cos\omega t,
\end{eqnarray}
where $\omega$ is the angular velocity of the rotation.

\begin{itemize}
\item[\bf a)] Find the Lagrangian expressed in terms of the coordinates $(\xi,\eta)$ and their time derivatives.
\item[\bf b)] Find the corresponding equations of motion for the two variables, and identify the Coriolis and centrifugal terms.
Compare with the standard expression for Newton's second law in a rotating reference frame, as can be found in introductory mechanics text books.
\end{itemize}
\begin{solution}
\begin{enumerate}[\bf a)]
\item As the mass moves in the horizontal plane, whether acted on by gravity or not, I can set the zero-potential in the plane. For the Lagrangian I then only have kinetic energy
\begin{gather*}
L = \half m \qty(\dot{x}^2 + \dot{y}^2)
\end{gather*}
where I find the coordinates
\begin{multicols}{2}
\small
\noindent
\begin{align*}
\dot{x}^2 =& \qty[\qty(\dot{\xi} -\eta \omega) \coswt - \qty(\xi\omega+\dot{\eta}) \sinwt]^2
\\
=& \coscoswt \qty(\dot{\xi}-\eta\omega)^2 + \sinsinwt\qty(\xi\omega + \dot{\eta})^2 
\\
&-2\coswt\sinwt \qty(\dot{\xi} -\eta \omega) \qty(\xi\omega+\dot{\eta})
\end{align*}
\begin{align*}
\dot{y}^2 =& \qty[ \qty(\dot{\xi} -\eta \omega) \sinwt + \qty(\xi\omega+\dot{\eta}) \coswt ]^2
\\
=&  \sinsinwt\qty(\dot{\xi}-\eta\omega)^2 +\coscoswt \qty(\xi\omega + \dot{\eta})^2 
\\
&+2\coswt\sinwt \qty(\dot{\xi} -\eta \omega) \qty(\xi\omega+\dot{\eta})
\end{align*}
\end{multicols}
resulting in a Lagrangian
\begin{gather*}
L = \half m \qty( \coscoswt \qty(\dot{\xi}-\eta\omega)^2 + \sinsinwt\qty(\xi\omega + \dot{\eta})^2 +  \sinsinwt\qty(\dot{\xi}-\eta\omega)^2 +\coscoswt \qty(\xi\omega + \dot{\eta})^2 )
\\
= \half m \qty( \qty(\dot{\xi}-\eta\omega)^2 +(\xi\omega + \dot{\eta})^2 )
\\
L = \half m \qty( \dot{\xi}^2 - 2 \dot{\xi}\eta\omega + \omega^2 \qty(\eta^2 + \xi^2) + \dot{\eta}^2 + 2\xi\omega\dot{\eta})
\end{gather*}
\item I assume the angular velocity of the rotating frame to be constant, so $\dot{\omega}=0$. Using Lagrange's equation on each of the coordinates
\begin{multicols}{2}
\noindent
\begin{gather*}
\dvtLpdv{\dot{\xi}} = \dvt{m \qty(\dot{\xi} -\omega\eta)} = m\qty(\ddot{\xi} -\omega\dot{\eta})
\\
\Lpdv{\xi} = m \qty( \omega^2\xi +\omega\dot{\eta})
\\
m\qty(\ddot{\xi} -\omega\dot{\eta}) - m \qty( \omega^2\xi + \omega\dot{\eta}) = 0
\\
\ddot{\xi} - \underbrace{\omega^2\xi}_{Centrifugal} - \underbrace{2\omega\dot{\eta}}_{Coriolis} = 0
\end{gather*}
\begin{gather*}
\dvtLpdv{\dot{\eta}} = \dvt{m\qty(\dot{\eta} + \omega\xi)} =m\qty( \ddot{\eta} + \omega\dot{\xi})
\\
\Lpdv{\eta} = m\qty(\omega^2\eta - \omega\dot{\xi})
\\
m\qty( \ddot{\eta} + \omega\dot{\xi}) -  m\qty(\omega^2\eta - \omega\dot{\xi}) = 0
\\
\ddot{\eta} - \underbrace{\omega^2\eta}_{Centrifugal}+ \underbrace{2\omega\dot{\xi}}_{Coriolis}= 0
\end{gather*}
\end{multicols}
I identify the centrifugal terms through the direct dependency of the coordinate it self in the e.o.m., and the Coriolis term as the term dependent on the time derivative of the <<other>> coordinate. This is also evident through the standard expression for Newton's second law in a rotating frame given as
\begin{gather*}
m\va{a} = \va{F} -2m\va{\omega}\cross\va{v} - m\dot{\va{\omega}}\cross \va{r} - m \va{\omega} \cross \qty(\va{\omega}\cross\va{r})
\\
\va{a} = -2\va{\omega}\cross\va{v} -\va{\omega} \cross \qty(\va{\omega}\cross\va{r})
\end{gather*}
where I canceled the external forces and the Euler force term (as these are assumed to be zero). The acceleration, velocity and position is given in the rotating reference frame $(\xi,\,\eta)$;
\begin{align*}
\va{a} = \mqty(\ddot{\xi}\\\ddot{\eta}\\0) \qc \va{v} =\mqty(\dot{\xi}\\\dot{\eta}\\0) \qc \va{r} = \mqty(x\\y\\0) \qc \va{\omega} = \mqty(0\\0\\\omega)
\end{align*} Writing out the cross products gives
\begin{gather*}
\mqty(\ddot{\xi}\\\ddot{\eta}\\0) = -2 \mqty(-\omega\dot{\eta}\\-\omega\dot{\xi}\\0) -\mqty(-\omega^2\xi\\-\omega^2\eta\\0)
\end{gather*}
which gives the same equations as I found from Lagrange's equation
\begin{align*}
\ddot{\xi} &= 2\omega\dot{\eta} + \omega^2 \xi & \ddot{\eta} = 2\omega\dot{\xi}+\omega^2\eta
\end{align*}
\end{enumerate}
\end{solution}
\end{exercise}


%%%%%%%%
\begin{exercise}[Midterm Exam 2008]\\
%%%%%%%%%%%
{\em The brachistochrone challenge.}\\
This is a classical problem in analytical mechanics. It was discussed by Galileo Galilei, who suggested a solution (but not the correct one), and studied the problem experimentally. In 1696 the problem was formulated as a challenge to the mathematicians at the time by Johann Bernoulli. He wrote in the journal {\em Acta Eruditorum}:\footnote{Johann Bernoulli, {\em Problema novum ad cujus solutionem Mathematici invitantur,} (A new problem to whose solution mathematicians are invited), Acta Eruditorum 18 (1696) 269.}

\begin{quote}
I, Johann Bernoulli, address the most brilliant mathematicians in the world. Nothing is more
attractive to intelligent people than an honest, challenging problem, whose possible solution
will bestow fame and remain as a lasting monument. Following the example set by Pascal,
Fermat, etc., I hope to gain the gratitude of the whole scientific community by placing
before the finest mathematicians of our time a problem which will test their methods and the
strength of their intellect. If someone communicates to me the solution of the proposed
problem, I shall publicly declare him worthy of praise.
\end{quote}

The problem he formulated was the following:

\begin{quote}
Given two points A and B in a vertical plane, what is the curve traced out by a point acted
on only by gravity, which starts at A and reaches B in the shortest time.
\end{quote}

Five solutions were obtained from scientist and mathematicians we are all acquainted with, Newton,
Jacob Bernoulli (the older brother of Johann), Leibniz and de L'H{\^o}pital, in addition to Johann himself. Johann Bernoulli gave a formulation of the problem where he could use an anology to Snell's law of refraction in optics to solve the problem.

%%%%%%%%%%%%
\begin{figure}[h]
\begin{center}
\includegraphics[height=5cm]{Brachistochrone.eps}
\end{center}
\caption{Illustration of the brachistochrone problem.}
\label{fig:brach}
\end{figure}
%%%%%%%%%%%%

Let us now rephrase the problem with a few more words: Assume a small body, named $P$ in Fig.~\ref{fig:brach}, moves in a vertical plane under the influence of gravity only. It leaves a point $A$ with zero velocity and follows (without friction) a given path in the plane which passes through a second point $B$, as shown in the figure. Assume the path between the two points $A$ and $B$ can be changed, while the points themselves stay fixed. For which path between the two points does the body spend the least time on the transit from point $A$ to point $B$? 

The challenge for you is the following: Find the solution to the brachistochrone problem by using the correspondence between the variational problem (finding the ``path of shortest time'') and the Lagrange equation, in the way discussed in the lectures. 

The body $P$ is to be treated as a point particle of mass $m$ and the path is represented by a function $y(x)$ with $x$ as the horizontal and $y$ as the vertical coordinate. The boundary conditions, which fix the positions of point $A$ and $B$, are specified as, $y(x_A)=y_A\,,\;y(x_B)=y_B$. 
To simplify the equations assume in the following that the initial coordinates are $x_A=y_A=0$.

\begin{itemize}
\item[\bf a)] Show that the time $T$  spent by the body on the way between $A$ and $B$ can be expressed as an integral of the form
\begin{equation}
T[y(x)]=\int_{x_A}^{x_B} L(y,y') dx, \label{eq:Time-action}
\end{equation}
with $y'=\frac{dy}{dx}$, and with 
\begin{equation}
L(y,y')=\sqrt{\frac{1+y'^2}{-2gy}}. \label{eq:Lagrangian_x_time}
\end{equation}
For the derivation, make use of energy conservation in the form $\frac{1}{2}m v^2+mgy=0$, with $y<0$.
\item[\bf b)] With $L(y,y')$ interpreted as a Lagrangian ($x$ then plays the role of $t$ in the usual formulation) the canonically conjugate momentum is $p=\frac{\partial L}{\partial y'}$ and the Hamiltonian is $H=py'-L$. Explain why $H$ is a constant of motion and use this fact to show that $y(x)$ satisfies a differential equation of the form
\begin{equation}
(1+y'^2)y=-k^2,
\label{eq:diff}
\end{equation}
with $k$ as a constant.
\item[\bf c)] Equation~(\ref{eq:diff}) has a solution which can be written on {\em parametric form} as
\begin{eqnarray}
&&x=\frac{1}{2} k^2(\theta-\sin\theta) \label{eq:parsolx},\\
&&y=\frac{1}{2} k^2(\cos\theta-1) \label{eq:parsoly},
\end{eqnarray}
where $\theta$ has been introduced as a curve parameter. Show that (\ref{eq:parsolx}) and (\ref{eq:parsoly}) constitute a solution of the differential equation (\ref{eq:diff}) by changing the variable from $x$ to $\theta$ in (\ref{eq:diff}), and by using the above expression for $y(\theta)$. In what way are the boundary conditions taken care of by this solution?

Relate $\theta$ to the boundary condition.

\item[\bf d)]  The curve $y(x)$ defined by the solution of the brachistochrone problem is a section of a {\em cycloid}, known for example as the curve traced out by a point on the periphery of a rolling wheel. Make a plot which shows the form of the curve. 
\item[\bf e)] Assume that point $B$ lies at the lowest point of the cycloid. Show that in this case the following relation has to be satisfied,  $y_B=-\frac{2}{\pi} \,x_B$. Calculate, for that situation, the time used by the body to reach point $B$ from $A$ and compare with the time used when the body instead follows a straight  line between the two points.
\end{itemize}
\begin{solution}
\begin{enumerate}[\bf a)]
\item To shown \cref{eq:Time-action} I use energy conservation, where the body start out with zero velocity at $y(x_A) = 0$, e.i. with zero potential and kinetic energy. This means that at some later point $x$ the sum of the energies is also zero with negative $y$ values
\begin{gather*}
\qq*{line element along $y(x)$} \dd s = \sqrt{\dd x^2 + \dd y^2} = \sqrt{1+\qty(\dv{y}{x})^2}\dd x
\\
\qq*{velocity along line element} v = \dv{s}{t} = \sqrt{1+\qty(\dv{y}{x})^2}\dv{x}{t}
\\
\half m v(x)^2 + mgy(x) = 0
\\
\Rightarrow \half m \qty[1 + \qty(\dv{y}{x})^2]\qty(\dv{x}{t})^2 + mgy = 0
\\
\sqrt{1 + \qty(\dv{y}{x})^2}\qty(\dv{x}{t})  = \sqrt{-2gy}
\\
\dv{t}{x} = \sqrt{\frac{1+y'^2}{-2gy}} \qq{where} y' = \dv{y}{x}
\end{gather*}
This is a separable equation for how the time changes with $x$. To find the time between two points I integrate between them giving
\begin{gather}
\int_0^T \dd t = \int_{x_A}^{x_B} \sqrt{\frac{1+y'^2}{-2gy}} \dd x  \label{eq:Timeeq}
\\
T[y(x)] = \int_{x_A}^{x_B} L(y,y') \dd x \notag
\end{gather}
as I wanted to show.
\item With \cref{eq:Lagrangian_x_time} interpreted as the Lagrangian with $x$ as the <<time>> coordinate I note that there are no explicit $x$ dependency, so the relation
\[ -\Lpdv{x} = \dv{H}{x} = 0 \]
indicates $H$ as a constant of motion. The Hamiltonian can also be thought of as the total energy, which I already used as a constant of motion in \textbf{a)}. To show \cref{eq:diff} I start by finding the Hamiltonian
\begin{gather*}
H = \Lpdv{y'} y' - L = \half \qty(\frac{1 + y'^2}{-2gy})^{-1/2}\qty(\frac{2y'}{-2gy})y' - L = E
\\
= \frac{y'^2}{-2gy} \qty(\frac{1 + y'^2}{-2gy})^{-1/2} - \qty(\frac{1 + y'^2}{-2gy})^{1/2} = E \quad \eval \cdot \qty(\frac{1 + y'^2}{-2gy})^{1/2}
\\
= \frac{y'^2}{-2gy} - \qty(\frac{1 + y'^2}{-2gy}) = E \qty(\frac{1 + y'^2}{-2gy})^{1/2}
\\
= \frac{1}{2gy}=E\qty(\frac{1 + y'^2}{-2gy})^{1/2} \qq{square both sides}
\\
\frac{1}{(2gy)^2} = E^2 \qty(\frac{1 + y'^2}{-2gy}) \qc  \qty(1+y'^2) = \frac{-1}{E^2gy} 
\\
\Rightarrow \qty(1+y'^2) y = -k^2 \qwhere k = \frac{1}{E\sqrt{g}}
\end{gather*}
\item To make the change of variables I express $y'$ as $y' = \dv{y}{\theta}\dv{x}{\theta}$.
\begin{gather*}
\dv{y}{\theta} = -\half k^2 \sint \qc \dv{x}{\theta} = \half k^2 \qty(1-\cost) \, \Rightarrow \, \dv{\theta}{x}= \frac{2}{k^2\qty(1-\cost)}
\end{gather*}
I insert these expressions in \cref{eq:diff} 
\begin{gather*}
1+y'^2 = 1 + \qty(\dv{y}{\theta}\dv{x}{\theta})^2 = 1 + \frac{\sinsint}{\qty(1-\cost)^2} = \qty((1-\cost)^2 + \sinsint) \frac{1}{(1-\cost)^2} 
\\
= \frac{2(1 - \cost)}{\qty(1-\cost)^2} = \frac{2}{1-\cost} 
\\
 \qq*{\cref{eq:diff} becomes} \frac{2}{1-\cost} \half k^2\qty(\cost -1) = -k^2
 \\
 \qq*{which reduces to} \frac{\cost -1}{1- \cost} = -1 \Rightarrow -1 = -1
\end{gather*}
where I have shown that the parametric form of the solution is consistent. For the boundary conditions I have that $x_A = y_A = 0$, which is consistent with $\theta_A = 0$. For the endpoint $B$ I have 
\begin{align*}
y_B &= \half k^2 \qty( \cos(\theta_B) -1) 
\\
x_B &= \half k^2 \qty( \theta_B - \sin(\theta_B))
\end{align*}
given a value for $\theta_B$ I can find the point $B = [x_B,\, y_B]$.
I have tried to solve the equations for $\theta_B$, which gave 
\begin{align*}
\qq*{from $y_B$:} \cos(\theta_B) &= \frac{2y_B}{k^2}+1
\\
\qq*{from $x_B$:}  \sin(\theta_B) &= \theta_B - \frac{2x_B}{k^2} \qor   \sin(\theta_B) =  \arccos(\frac{2y_B}{k^2}+1) - \frac{2x_B}{k^2}
\\
\frac{\sin(\theta_B)}{\cos(\theta_B)} = \tan(\theta_B) &= \frac{\theta_B - \frac{2x_B}{k^2}}{\frac{2y_B}{k^2}+1}
\end{align*}
but I have not been able to use these expressions correctly in the following where I have plotted the curve, so I might have made a mistake somewhere.

\item As discussed, I have not been able to find a value for $\theta_B$ given some point $B$ that makes sense in the plot. I have therefore chosen a value for $\theta_B = \frac{3\pi}{2}$ and plotted the curve using the parameterized equations (\ref{eq:parsolx}) and (\ref{eq:parsoly}).
\fig[Brachistochrone curve from parameterized equations (\ref{eq:parsolx}) and (\ref{eq:parsoly}). $\theta_A = 0$, $\theta_B = \frac{3\pi}{2}$.]{brachistochrone.pdf}

\item If $B$ is the lowest point of the cycloid then $\dv{y}{\theta} = 0 $ at $\theta_B$. So
\[ \dv{y}{\theta} = -\half k^2\sin(\theta_B) = 0 \Rightarrow \theta_B = \pi \] 
using this in the parameterized equations gives
\begin{align*}
y_B &= \half k^2 \qty(\cos(\pi)-1) = -k^2
\\
x_B &= \half k^2 \qty( \pi - \sin(\pi)) = \half k^2 \pi
\\
\frac{y_B}{x_B} &= \frac{-k^2}{\half k^2 \pi } \Rightarrow y_B = \frac{-2}{\pi}x_B.
\end{align*}
To find the time between the two points I use \cref{eq:Timeeq} with the change of variable from $x$ to $\theta$
\begin{gather*}
1+y'^2 = \frac{2}{1-\cost} \qc -2gy = gk^2\qty(1-\cost)
\\
\qq{so} \frac{1+y'^2}{-2gy} = \frac{2}{gk^2\qty(1-\cost)^2} \qand \dd x = \half k^2\qty(1-\cost)\dd \theta
\\
\qq*{the integrand becomes} \sqrt{\frac{2}{gk^2\qty(1-\cost)^2}} \half k^2\qty(1-\cost)\dd \theta = \sqrt{\frac{2}{gk^2}}\half k^2 \dd \theta = \frac{k}{\sqrt{2g}} \dd \theta
\\
\qq*{integrating in $t\in[0,T_B]$ and $\theta\in[0,\pi]$} \int_0^{T_B} \dd t = T_B = \int_0^{\pi} \frac{k}{\sqrt{2g}} \dd \theta
\\
\qq*{finally giving the time} T_B = \frac{k\pi}{\sqrt{2g}} \approx 0.71 k
\end{gather*}
For the straight line I have $y = ax + b$, with the boundary conditions together with the parameterized equations implying $a = -\frac{2}{\pi}$, $b= 0$ and the endpoint $x_B = \frac{\pi}{2}k^2$. With this $y(x)$ the Lagrangian becomes
\[ L = \sqrt{\frac{1+y'^2}{-2gy}} = \sqrt{\frac{1+\frac{4}{\pi^2}}{\frac{4g}{\pi}x}} = \sqrt{\frac{\pi^2+4}{\pi 4 g x}}\]
so I can find the time between $A$ and $B$ on a straight line
\begin{align*}
T_S &= \int_{x_A}^{x_B} L(y,y') \dd x = \int_{x_A}^{x_B} \sqrt{\frac{\pi^2+4}{\pi 4 g}}x^{-1/2} \dd x
\\
&= 2 \sqrt{\frac{\pi^2+4}{\pi 4 g}} x^{1/2} \eval_{x_A = 0}^{x_B=\frac{\pi k^2}{2}} = 2 \sqrt{\frac{\pi^2+4}{\pi 4 g}} \sqrt{\frac{\pi}{2}}k \approx 0.84 k.
\end{align*}
The fraction of the two times
\[ \frac{T_B}{T_S} = \frac{0.71 k}{0.84 k} \approx 0.845 \Rightarrow T_B = 0.845 T_S\]
so the time used on the brachistochrone curve is approximately $85\%$ of the time used on the straight line.
\end{enumerate}
\end{solution}
\end{exercise}


\end{document}

